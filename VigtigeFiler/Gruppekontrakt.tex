\documentclass[a4paper,12pt]{article}
\usepackage[utf8]{inputenc}
\usepackage[danish]{babel}
\usepackage{lipsum}

\title{Gruppekontrakt}
\author{}
\date{}

\begin{document}

\maketitle

\section*{Forventninger til samarbejdet}
Nøgleordet i vores samarbejde er engagement i vores fælles mål, hvor alle bidrager med deres unikke perspektiv og færdigheder.

\section*{Rammer for arbejdet}
\subsection*{Hvor og hvornår holder vi gruppemøde:}
\begin{itemize}
    \item Tirsdage i Shannon før eller efter vejledermøde i forbindelse med skemalagt projekt.
    \item Fredage er der arbejdstid fra 9 til ca. 14 eller alt efter behov.
    \item Efter behov aftales indbyrdes.
\end{itemize}

\subsection*{Hvornår og hvor holder vi vejledermøde:}
\begin{itemize}
    \item Tirsdage kl. 10(?) i Shannon.
\end{itemize}

\section*{Rollefordeling}
\begin{itemize}
    \item Mødeleder: Sanne
    \item Referent: 
\end{itemize}

\section*{Kommunikation i gruppen}
\subsection*{Beskeder:}
Facebook gruppe: \textbf{"Proj3"}. Giv en reaktion som minimum respons.

\subsection*{Fravær/ Afbud}
\begin{itemize}
    \item Hvordan og hvor sent melder man afbud? \\
    Over Facebook gruppen. Dagen før hvis der er tale om ikke akutte hændelser (ferie, andre aftaler etc.). På dagen melder man kun afbud ved sygdom eller tilskadekomst.
    \item Hvad sker der hvis man ikke melder afbud eller kommer for sent til et aftalt møde? \\
    Så skylder man kvajebajer.
\end{itemize}

\section*{Konfliktløsning}
\subsection*{Hvilke problemer kan der forekomme?}
Vores forskellige profiler (mht. indsigt profiler) kan give anledning til uoverensstemmelse mht. måden at samarbejde på.

\subsection*{Hvordan sikrer vi, at aftalerne bliver overholdt?}
Hvis aftaler ikke overholdes, skylder personen kvajebajer.

\section*{Beslutningstagen}
\subsection*{Hvem træffer beslutningerne?}
Større beslutninger foregår ved konsensus i gruppen, hvor beslutningen tages ud fra flertallet.

\section*{Respekt og samarbejde}
Vi bestræber os på følgende:
\begin{itemize}
    \item At have en god omgangstone samt tale til og med respekt for hinanden.
    \item At være åbne og anerkendende i samarbejdet og omkring hinandens ideer og input.
    \item At være indforståede med at vi alle er forskellige og har forskellige styrker og svagheder.
\end{itemize}

\section*{Underskrifter}

\vspace{1cm}

\noindent\makebox[\textwidth]{\hrulefill}\\
Filip Fontao

\vspace{1cm}

\noindent\makebox[\textwidth]{\hrulefill}\\
Jakob Alsina Lund

\vspace{1cm}

\noindent\makebox[\textwidth]{\hrulefill}\\
Kevin Nguyen

\vspace{1cm}

\noindent\makebox[\textwidth]{\hrulefill}\\
Lasse Fink

\vspace{1cm}

\noindent\makebox[\textwidth]{\hrulefill}\\
Lucas Stilborg

\vspace{1cm}

\noindent\makebox[\textwidth]{\hrulefill}\\
Magnus Hvidsten Christiansen

\vspace{1cm}

\noindent\makebox[\textwidth]{\hrulefill}\\
Rene Sand Schumacher

\vspace{1cm}

\noindent\makebox[\textwidth]{\hrulefill}\\
Sanne Yding Feddersen

\vspace{1cm}

\noindent\makebox[\textwidth]{\hrulefill}\\
Simon Bøjgård Nowack

\end{document}
