\documentclass{article}

% Packages
\usepackage[utf8]{inputenc}
\usepackage[T1]{fontenc}
\usepackage{geometry}
\usepackage{titlesec}

% Page setup
\geometry{a4paper, margin=2.5cm}
\titleformat{\section}{\normalfont\Large\bfseries}{\thesection}{1em}{}
\titleformat{\subsection}{\normalfont\large\bfseries}{\thesubsection}{1em}{}

% Title
\title{Problemformulering}
\author{Group 4}
\date{\today}

\begin{document}

\maketitle

\section{Problemformulering}
% Add your introduction here
Mange mennesker oplever udfordringer med at optimere deres søvnkvalitet, selvom de overholder anbefalingerne om 7-9 timers søvn pr. nat. Søvnkvalitet påvirkes af en række faktorer, såsom søvnstadie, puls, bevægelser, og støj i omgivelserne, men den gennemsnitlige person har begrænset indsigt i disse parametre og deres indflydelse på søvnens effektivitet. Dette fører ofte til en situation, hvor selv tilstrækkelig søvn i antal timer ikke nødvendigvis resulterer i optimal hvile eller restitution.

Formålet med dette projekt er at udvikle SmartSleep, et system, der ved hjælp af sensorer kan overvåge og analysere en brugers søvn i realtid. Ved at anvende data fra bevægelsessensorer, mikrofoner, pulsmålere og EEG-målinger fra hjernen, vil systemet være i stand til at identificere søvnmønstre, herunder urolig søvn, snorken og pulsvariationer, samt give en detaljeret evaluering af søvnens kvalitet. Systemet vil præsentere en brugervenlig søvnrapport på en touchskærm tilknyttet en Raspberry Pi, og vil desuden fungere som et smart vækkeur, der kan tilpasses brugerens individuelle behov. Gennem disse funktionaliteter søger projektet at forbedre den enkeltes evne til at optimere deres søvnvaner og dermed opnå en højere livskvalitet.

Denne problemstilling er relevant, da en stor del af befolkningen undervurderer søvnens betydning for sundhed og præstation, og vi sigter derfor mod at skabe et værktøj, der kan give brugeren den nødvendige indsigt til at forbedre søvnkvaliteten.

\end{document}